\textbf{\large Problem Description}

Toby is analysing the railway system of his city, and he has noticed that it can
be modeled like a directed graph where the stations are the vertices and the edges
are the rails that connect them.

Toby got a lot queries for you and for each query he wants to know how many
stations are reachable if he is actually at the station $S$, by reachable he
means that can go from $S$ to an station $T$ using one or more rails.

\textbf{\large Input Format}

The input contains two numbers $N$, $M$ denoting number of stations and rails
respectively. Each one of the following $M$ lines contains two integers $(X, Y)$
denoting that there is a connection from $X$ to $Y$, then one integer $Q$
denoting the number of queries that Toby is going to give to you, in the next $Q$
lines an integer $S$ is given, that is the city which toby wants to know how
many stations are reachable from it.

\textbf{\large Note:}

You can assume that $X$ and $Y$ are different.

$1 \le N, Q \le 1000$ \\
$0 \le M \le min((N * (N - 1)), 2000) $ % TODO: change value of M to 1000

\textbf{\large Output Format}

For each query print the number of stations reachable from $S$.

\textbf{\large Sample Input}

\begin{verbatim}
7 9
1 2
2 4
4 3
3 1
4 5
1 5
7 2
7 6
6 7
3
6
5
2
\end{verbatim}

\textbf{\large Sample Output}

\begin{verbatim}
6
0
4
\end{verbatim}

\newpage
