\documentclass[11pt]{article}
    \usepackage{graphicx}
    \usepackage{parskip}

    \usepackage{geometry}
    \geometry{
     a4paper,
    % total={210mm,297mm},
     left=1in,
     right=1in,
     top=1in,
     bottom=1in
    }

    \usepackage{fancyhdr}
    \lhead{\includegraphics[height=1cm]{logo.png}}
    \chead{\raisebox{0.7mm}{2018 UTP Internal Programming Contest}}
    \rhead{\includegraphics[height=1cm]{icpc_logo.png}}
    \pagestyle{fancy}

    \setlength{\headheight}{27pt}

    \usepackage{csquotes}

    \begin{document}
        \pagenumbering{gobble}
        \pagenumbering{arabic}
        \setcounter{page}{1}

        \begin{center}
{\LARGE 2018 UTP Internal Programming Contest}\\
{\LARGE Contest Information}
\end{center}

%\section*{Ethic}
%Contestants shall respect their opponents. Contestants shall abide the rules. Contestants shall compete for the honor of themselves, of their teams, and of their schools. Contestants shall not compromise the reputation of the society of competitive programming.

\section*{Rules}
Contestants will be disqualified if they violate any one of the following rules.
\begin{enumerate}
\item No machine-readable materials (e.g., source codes, templates, etc.) are allowed. However, paper-based materials, such as textbooks, dictionaries, printed notes, etc., are allowed.
\item Contestants are only allowed to contact his/her teammates during the contest. Contestants shall not discuss with his/her coach and other teams.
\item Contestants shall only access the internet for downloading the problem description, submitting source codes, requesting problem clarification and checking the scoreboard. Any other type of internet access is prohibited.
\item A team shall not simultaneously use more than one computer to write programs during the contest. Contestant shall not use any other type of electronic devices, except extra monitors and printers.
\item All malicious actions interfering the contest are prohibited.
\end{enumerate}

\section*{Scoring and Ranking}
\begin{enumerate}
\item Disqualified teams will be removed from the ranking.
\item Only C, C++, Java, Python are provided in this contest. The judge system only accepts programs which can be properly compiled and executed. A problem is solved if the submitted program terminates and outputs correctly in time. The responses of the judge system are listed as follows.
{	\setlength{\parskip}{1pt}
	\begin{itemize}
    \item CE: The program cannot be properly compiled or executed.
    \item TLE: The program uses too much time.
    \item RE: Run-time error. The program cannot terminate normally.
    \item MLE: The program uses too much memory.
    \item WA: The output is incorrect.
    \item AC: The program is accepted by the judge system, and the problem is solved.
    \end{itemize}
}
\item Teams are ranked according to the most problems solved. Teams who solve the same number of problems are ranked by least total time. The total time is the sum of the time consumed for each problem solved. The time consumed for a solved problem is the time elapsed from the beginning of the contest to the submittal of the accepted run plus 20 penalty minutes for every rejected run for that problem regardless of submittal time. There is no time consumed for a problem that is not solved.

\item Breaking ties in ranking, if necessary, is according to the following order:
{	\setlength{\parskip}{1pt}
    \begin{enumerate}
  	\item the less total number of submissions of the solved problems;
  	\item the shorter elapsed time of the first solved problem;
  	\item the shorter elapsed time of the second solved problem, and so on.
  	\end{enumerate}
}
\end{enumerate}

\newpage


        
        \begin{center}
            {\LARGE Problem A}\\
            {\Large Toby And Love}\\
            {Time limit: 2 second}\\
            {Memory limit: 256 megabytes}
        \end{center}\textbf{\large Problem Description}

Toby has a lot of messages from his beautiful girl, today Toby is wondering about
the amount of love that each message has.

For each message Toby wants to know how many times the word ``love'' appears.
Can you help this little pet?

\textbf{\large Input Format}

The first line contains a single integer $(1 \le N \le 100)$ the amount of messages.
The next $N$ lines contain a single message $S$, the length of $S$ does not exceed 1000 characters
and only contains lowercase Latin letters.

\textbf{\large Output Format}

For every message $S$ print a single integer.

\textbf{\large Sample Input}

\begin{verbatim}
4
iinlovewithyou
lovelovelove
mylov
tobyiloveyousomuchyouaremytruelove
\end{verbatim}

\textbf{\large Sample Output}

\begin{verbatim}
1
3
0
2
\end{verbatim}

\newpage

        \begin{center}
            {\LARGE Problem B}\\
            {\Large Toby And Coins}\\
            {Time limit: 2 second}\\
            {Memory limit: 256 megabytes}
        \end{center}\textbf{\large Problem Description}

Toby is going to buy a machine to send love letters to his girlfriend, the machine
costs $N$ pesos. Toby works very hard and he has a lot of money, in fact, he can
pay the machine with any combination of coins!

Toby wants to know what is the \textbf{minimum} number of coins he needs to buy the machine.

In the Toby's city there are coins with the following values:

$\{1, 2, 5, 10, 20\}$


\textbf{\large Input Format}

The first line contains an integer $1 < T < 100$ denoting the number of test cases. For each
one of the next $T$ lines, there is an integer $1 < C < 10000$ denoting the cost of the machine.

\textbf{\large Output Format}

For each test case, print the minumum number of coins that Toby needs in order
to buy the machine.

\textbf{\large Sample Input}

\begin{verbatim}
3
15
8
22
\end{verbatim}

\textbf{\large Sample Output}

\begin{verbatim}
2
3
2
\end{verbatim}

\newpage
        \begin{center}
            {\LARGE Problem C}\\
            {\Large Toby And Stars}\\
            {Time limit: 2 second}\\
            {Memory limit: 256 megabytes}
        \end{center}\textbf{\large Problem Description}

Toby is looking at the sky and he found a lot of beautiful stars, he is
wondering what is the minimum distance between any pair of stars. Could you
help him ?

Note: You can safetly asume that the stars are in a 2D plane.

\textbf{\large Input Format}

The input contains a number $N$ denoting the total number of stars. Each one of
the following $N$ lines, contains a pair of integers denoting the position of
one star.

$2 \le N \le 500$

The coordinates of each star are between 0 and 1000

\textbf{\large Output Format}

Print the minimum distance between any pair of stars. The answer is considered
valid if the difference with the correct value is less than 1e-4

\textbf{\large Sample Input}

\begin{verbatim}
5
434 155
8 412
100 816
301 762
312 506
\end{verbatim}

\textbf{\large Sample Output}

\begin{verbatim}
208.127364851
\end{verbatim}

\newpage
        \begin{center}
            {\LARGE Problem D}\\
            {\Large Toby And Primes}\\
            {Time limit: 2 second}\\
            {Memory limit: 256 megabytes}
        \end{center}\textbf{\large Problem Description}

Toby loves prime numbers and today he has the next challenge for you.
You have an integer $N$ ($N$ has between 1 and 4 digits), is possible to reorder the
digits of the number in such a way that one of the resulting numbers is a prime number?
Note: The resulting number can't have leading zeros.

\textbf{\large Input Format}

The first line contains an integer $T$, denoting the number of test cases, in each of
the next $T$ lines there is an integer $N$.

\textbf{\large Output Format}

For each test case you have to print ``YES'' if is possible to reorder the digits and make
a prime number or print ``NO'' otherwise.

\textbf{\large Sample Input}

\begin{verbatim}
6
1
7
712
209
24
1798
\end{verbatim}

\textbf{\large Sample Output}

\begin{verbatim}
NO
YES
YES
NO
NO
YES
\end{verbatim}

\textbf{\large Explanation:}

For the fourth case 209 can be reorder like this (029, 092, 209, 290, 902, 920), 029 is prime,
but is not a valid number because has leading zeros, in the valid numbers (209, 290, 902, 920).
there are no primes, so the answer is ``NO''

For the fith case 24 only has two possible numbers (24, 42) and there are no primes, so the
answer is ``NO''.

\newpage
        \begin{center}
            {\LARGE Problem E}\\
            {\Large Toby And Numbers}\\
            {Time limit: 2 second}\\
            {Memory limit: 256 megabytes}
        \end{center}\textbf{\large Problem Description}

Toby just invented a game with numbers, initially you have a list of numbers
and at any moment you can do the following operation:

\begin{itemize}
    \item Take two different numbers and replace the greater one with the absolute
    difference of those numbers.
    \item The game ends when all the numbers are equal.
\end{itemize}

It can be shown that the game always ends and no matter how you play, the result
will be always the same.

Help toby to determine what is the result for several instances of the game.

\textbf{\large Input Format}

The input begins with an integer $T$ denoting the number of test cases. For each
test case, there is a number $N$ indicating how many numbers are in the current case,
followed by $N$ positive integers.

$1 \le N \le 100$

Each one of the $N$ numbers is between 1 and 10000

\textbf{\large Output Format}

Print one integer for each test case. Note that at end of each game all the
numbers are equal, for this reason you only need to print it once, no matter what
is the size of the input list.

\textbf{\large Sample Input}

\begin{verbatim}
2
3
3 6 9
5
2 8 10 16 36
\end{verbatim}

\textbf{\large Sample Output}

\begin{verbatim}
3
2
\end{verbatim}

\newpage
        \begin{center}
            {\LARGE Problem F}\\
            {\Large Toby And Sheeps}\\
            {Time limit: 2 second}\\
            {Memory limit: 256 megabytes}
        \end{center}\textbf{\large Problem Description}

Toby is now a shepherd and has a flock of sheeps all of them in a row, but alas there are wolves too, luckly our pet is pretty relaxed and will enter on panic only if when he sight a subsegment of his flock the number of wolves is greater than the number of sheeps.

Toby has few questions for you, as he is too lazy, he wants to ask you if for an given interval  (defined by $[L, R]$) there are more wolves than sheeps, you have to answer Yes or No.

Note: A sheep is represented by a '1' and a wolf is represented by a '0'.

\textbf{\large Input Format}

There are several test cases.
Each test case starts by a number $N$ denoting the number animals in the flock (sheeps and wolves), next you have to read $N$ numbers (0 or 1); followed by a number $Q$, the number of queries that Toby got for you, then $Q$ lines are given and you will have to read $L$ and $R$ denoting the begin and the end of the subsegment that Toby wants to check.

$1 \le N, Q \le 100000$ \\
$1 \le L \le R \le N$

\textbf{\large Output Format}

For each query you have to output ``Yes'' if the numbers of wolves is greater than the numbers of sheeps, and ``No'' otherwise. Without quotes.

\textbf{\large Sample Input}

\begin{verbatim}
4
1 0 1 1
3
1 2
2 2
1 4
\end{verbatim}

\textbf{\large Sample Output}

\begin{verbatim}
No
Yes
No
\end{verbatim}

\newpage

        \begin{center}
            {\LARGE Problem G}\\
            {\Large Toby And Candies}\\
            {Time limit: 2 second}\\
            {Memory limit: 256 megabytes}
        \end{center}\textbf{\large Problem Description}

Toby has $D$ candies and he wants to know the number of ways he can distribute
those candies between $K$ friends. Note that for a particular distribution of the
candies, some people could get no candies at all.

\textbf{\large Input Format}

The input starts with a number $T$ indicating the number of test cases. For each one
of the following $T$ lines, there are two integers $D$ and $K$ denoting the number
of candies and the number of friends respectively.

$1 \le T, D, K \le 10$

\textbf{\large Output Format}

For each test case print the answer to the problem in a new line

\textbf{\large Sample Input}

\begin{verbatim}
1
3 2
\end{verbatim}

\textbf{\large Sample Output}

\begin{verbatim}
4
\end{verbatim}

\textbf{\large Explanation}

the four ways to distribute the candies among 2 people are:

\begin{verbatim}
3, 0
2, 1
1, 2
0, 3
\end{verbatim}

\newpage
        \begin{center}
            {\LARGE Problem H}\\
            {\Large Toby And Sherlock}\\
            {Time limit: 2 second}\\
            {Memory limit: 256 megabytes}
        \end{center}\textbf{\large Problem Description}

Toby and his friend Sherlock are playing a game with stones, the rules are:
\begin{itemize}
\item Initially there are N stones.
\item  They play in turns.
\item  Toby plays first.
\item  Each player could take between 1 and 5 stones in one turn.
\item  If one player has no stones to take, he loses the game.
\end{itemize}

Help Toby to determine who will win for a several number of stones. Take into account that both players will play optimally.

\textbf{\large Input Format}

The input begins with a number $T$ denoting the number of test cases. For each
test case a positive number $N$ is given and indicates the number of stones.

$ 1 \le T \le 1000 $ \\
$ 1 \le N \le 1000 $

\textbf{\large Output Format}

For each test case print ``Toby'' if Toby wins the game, print ``Sherlock'' otherwise.

\textbf{\large Sample Input}

\begin{verbatim}
2
4
7
\end{verbatim}

\textbf{\large Sample Output}

\begin{verbatim}
Toby
Toby
\end{verbatim}

\newpage

        \begin{center}
            {\LARGE Problem I}\\
            {\Large Toby And Addition}\\
            {Time limit: 2 second}\\
            {Memory limit: 256 megabytes}
        \end{center}\textbf{\large Problem Description}

Toby lives in Bitland where things are a little weird,
by example 4 + 5 = 3 woow!. Definitely the addition is not like we know.
Toby like a regular citizen knows very well the process for addition in Bitland.
He say, ``Is easy, just count the total amount of bits turn on between numbers''

\textbf{\large Example}

$4 + 5 = 3$ \\
$4 -> 100$: there is 1 bit turn on \\
$5 -> 101$: there are 2 bits turn on \\

So the answer is 1 + 2 = 3

For this challenge you need to learn how Bitland's citizens add two number.

\textbf{\large Input Format}

The first line contains an integer $T$, $(T \le 1000)$, the number of test cases,
each test case contains two integers $a$ and $b$ $(0 \le a,b \le 10^{18})$.

\textbf{\large Note:}

to store a and b you need more than 32 bits, so watch out! :P

\textbf{\large Output Format}

For each test case print a single integer with the answer.

\textbf{\large Sample Input}

\begin{verbatim}
5
1 1
8 64
4 5
66666666 33333333
576460752303423487 576460752303423487
\end{verbatim}

\textbf{\large Sample Output}

\begin{verbatim}
2
2
3
26
118
\end{verbatim}

\newpage

        \begin{center}
            {\LARGE Problem J}\\
            {\Large Toby And The Mall}\\
            {Time limit: 2 second}\\
            {Memory limit: 256 megabytes}
        \end{center}\textbf{\large Problem Description}

Toby is lost in a mall and now is looking for the exit, the mall can be modeled
as a matrix in 2D of $N$ rows and $M$ columns, Toby is located at the upper-left
corner and the exit is located at the lower-right corner, Toby only has 2
kinds of moves, he can move one step right or one step down to the adjacent
cells. Each cell has a price that must be payed in order to stay there.
Can you help to our pet to find the exit paying the lowest possible price?.

\textbf{\large Input Format}

The input contains two numbers $N$, $M$ denoting number of rows and columns respectively.
Each one of the following $N$ lines, contains $M$ integers denoting the price that
Toby must pay to stay in the $(i-th, j-th)$ cell.

$1 \le N, M \le 1000$

Each element in the matrix is between 1 and 1000.

\textbf{\large Output Format}

Print the minimum price that Toby has to pay to reach the exit.

\textbf{\large Sample Input}

\begin{verbatim}
4 5
1 1 1 5 6
2 2 1 8 1
2 1 8 1 2
1 1 1 1 1
\end{verbatim}

\textbf{\large Sample Output}

\begin{verbatim}
9
\end{verbatim}

\newpage
        \begin{center}
            {\LARGE Problem K}\\
            {\Large Toby And The Railway Stations}\\
            {Time limit: 2 second}\\
            {Memory limit: 256 megabytes}
        \end{center}\textbf{\large Problem Description}

Toby is analysing the railway system of his city, and he has noticed that it can
be modeled like a directed graph where the stations are the vertices and the edges
are the rails that connect them.

Toby got a lot queries for you and for each query he wants to know how many
stations are reachable if he is actually at the station $S$, by reachable he
means that can go from $S$ to an station $T$ using one or more rails.

\textbf{\large Input Format}

The input contains two numbers $N$, $M$ denoting number of stations and rails
respectively. Each one of the following $M$ lines contains two integers $(X, Y)$
denoting that there is a connection from $X$ to $Y$, then one integer $Q$
denoting the number of queries that Toby is going to give to you, in the next $Q$
lines an integer $S$ is given, that is the city which toby wants to know how
many stations are reachable from it.

\textbf{\large Note:}

You can assume that $X$ and $Y$ are different.

$1 \le N, Q \le 1000$ \\
$0 \le M \le (N * (N - 1)) / 2$ % TODO: change value of M to 1000

\textbf{\large Output Format}

For each query print the number of stations reachable from $S$.

\textbf{\large Sample Input}

\begin{verbatim}
7 9
1 2
2 4
4 3
3 1
4 5
1 5
7 2
7 6
6 7
3
6
5
2
\end{verbatim}

\textbf{\large Sample Output}

\begin{verbatim}
6
0
4
\end{verbatim}

\newpage
    \end{document}
