\documentclass[11pt]{article}
    \usepackage{graphicx}
    \usepackage{parskip}

    \usepackage{geometry}
    \geometry{
     a4paper,
    % total={210mm,297mm},
     left=1in,
     right=1in,
     top=1in,
     bottom=1in
    }

    \usepackage{fancyhdr}
    \lhead{\includegraphics[height=1cm]{logo.png}}
    \chead{\raisebox{0.7mm}{2018 UTP Internal Programming Contest}}
    \rhead{\includegraphics[height=1cm]{icpc_logo.png}}
    \pagestyle{fancy}

    \setlength{\headheight}{27pt}

    \usepackage{csquotes}

    \begin{document}
        \pagenumbering{gobble}
        \pagenumbering{arabic}
        \setcounter{page}{1}

        \begin{center}
{\LARGE 2018 UTP Internal Programming Contest}\\
{\LARGE Contest Information}
\end{center}

%\section*{Ethic}
%Contestants shall respect their opponents. Contestants shall abide the rules. Contestants shall compete for the honor of themselves, of their teams, and of their schools. Contestants shall not compromise the reputation of the society of competitive programming.

\section*{Rules}
Contestants will be disqualified if they violate any one of the following rules.
\begin{enumerate}
\item No machine-readable materials (e.g., source codes, templates, etc.) are allowed. However, paper-based materials, such as textbooks, dictionaries, printed notes, etc., are allowed.
\item Contestants are only allowed to contact his/her teammates during the contest. Contestants shall not discuss with his/her coach and other teams.
\item Contestants shall only access the internet for downloading the problem description, submitting source codes, requesting problem clarification and checking the scoreboard. Any other type of internet access is prohibited.
\item A team shall not simultaneously use more than one computer to write programs during the contest. Contestant shall not use any other type of electronic devices, except extra monitors and printers.
\item All malicious actions interfering the contest are prohibited.
\end{enumerate}

\section*{Scoring and Ranking}
\begin{enumerate}
\item Disqualified teams will be removed from the ranking.
\item Only C, C++, Java, Python are provided in this contest. The judge system only accepts programs which can be properly compiled and executed. A problem is solved if the submitted program terminates and outputs correctly in time. The responses of the judge system are listed as follows.
{	\setlength{\parskip}{1pt}
	\begin{itemize}
    \item CE: The program cannot be properly compiled or executed.
    \item TLE: The program uses too much time.
    \item RE: Run-time error. The program cannot terminate normally.
    \item MLE: The program uses too much memory.
    \item WA: The output is incorrect.
    \item AC: The program is accepted by the judge system, and the problem is solved.
    \end{itemize}
}
\item Teams are ranked according to the most problems solved. Teams who solve the same number of problems are ranked by least total time. The total time is the sum of the time consumed for each problem solved. The time consumed for a solved problem is the time elapsed from the beginning of the contest to the submittal of the accepted run plus 20 penalty minutes for every rejected run for that problem regardless of submittal time. There is no time consumed for a problem that is not solved.

\item Breaking ties in ranking, if necessary, is according to the following order:
{	\setlength{\parskip}{1pt}
    \begin{enumerate}
  	\item the less total number of submissions of the solved problems;
  	\item the shorter elapsed time of the first solved problem;
  	\item the shorter elapsed time of the second solved problem, and so on.
  	\end{enumerate}
}
\end{enumerate}

\newpage


        
        \begin{center}
            {\LARGE Problem A}\\
            {\Large Toby And Love}\\
            {Time limit: 2 second}\\
            {Memory limit: 256 megabytes}
        \end{center}\textbf{\large Problem Description}

Toby has a lot of messages from his beautiful girl, today Toby is wondering about
the amount of love that each message has.

For each message Toby wants to know how many times the word ``love'' appears.
Can you help this little pet?

\textbf{\large Input Format}

The first line contains a single integer $(1 \le N \le 100)$ the amount of messages.
The next $N$ lines contain a single message $S$, the length of $S$ does not exceed 1000 characters
and only contains lowercase Latin letters.

\textbf{\large Output Format}

For every message $S$ print a single integer.

\textbf{\large Sample Input}

\begin{verbatim}
4
iinlovewithyou
lovelovelove
mylov
tobyiloveyousomuchyouaremytruelove
\end{verbatim}

\textbf{\large Sample Output}

\begin{verbatim}
1
3
0
2
\end{verbatim}

\newpage

        \begin{center}
            {\LARGE Problem B}\\
            {\Large Toby And Coins}\\
            {Time limit: 2 second}\\
            {Memory limit: 256 megabytes}
        \end{center}\textbf{\large Problem Description}

Toby is going to buy a machine to send love letters to his girlfriend, the machine
costs $N$ pesos. Toby works very hard and he has a lot of money, in fact, he can
pay the machine with any combination of coins!

Toby wants to know what is the \textbf{minimum} number of coins he needs to buy the machine.

In the Toby's city there are coins with the following values:

$\{1, 2, 5, 10, 20\}$


\textbf{\large Input Format}

The first line contains an integer $1 < T < 100$ denoting the number of test cases. For each
one of the next $T$ lines, there is an integer $1 < C < 10000$ denoting the cost of the machine.

\textbf{\large Output Format}

For each test case, print the minumum number of coins that Toby needs in order
to buy the machine.

\textbf{\large Sample Input}

\begin{verbatim}
3
15
8
22
\end{verbatim}

\textbf{\large Sample Output}

\begin{verbatim}
2
3
2
\end{verbatim}

\newpage
        \begin{center}
            {\LARGE Problem C}\\
            {\Large Toby And Stars}\\
            {Time limit: 2 second}\\
            {Memory limit: 256 megabytes}
        \end{center}\textbf{\large Problem Description}

Toby is looking at the sky and he found a lot of beautiful stars, he is
wondering what is the minimum distance between any pair of stars. Could you
help him ?

Note: You can safetly asume that the stars are in a 2D plane.

\textbf{\large Input Format}

The input contains a number $N$ denoting the total number of stars. Each one of
the following $N$ lines, contains a pair of integers denoting the position of
one star.

$2 \le N \le 500$

The coordinates of each star are between 0 and 1000

\textbf{\large Output Format}

Print the minimum distance between any pair of stars. The answer is considered
valid if the difference with the correct value is less than 1e-4

\textbf{\large Sample Input}

\begin{verbatim}
5
434 155
8 412
100 816
301 762
312 506
\end{verbatim}

\textbf{\large Sample Output}

\begin{verbatim}
208.127364851
\end{verbatim}

\newpage
        \begin{center}
            {\LARGE Problem D}\\
            {\Large Toby And Primes}\\
            {Time limit: 2 second}\\
            {Memory limit: 256 megabytes}
        \end{center}\textbf{\large Problem Description}

Toby loves prime numbers and today he has the next challenge for you.
You have an integer $N$ ($N$ has between 1 and 4 digits), is possible to reorder the
digits of the number in such a way that one of the resulting numbers is a prime number?
Note: The resulting number can't have leading zeros.

\textbf{\large Input Format}

The first line contains an integer $T$, denoting the number of test cases, in each of
the next $T$ lines there is an integer $N$.

\textbf{\large Output Format}

For each test case you have to print ``YES'' if is possible to reorder the digits and make
a prime number or print ``NO'' otherwise.

\textbf{\large Sample Input}

\begin{verbatim}
6
1
7
712
209
24
1798
\end{verbatim}

\textbf{\large Sample Output}

\begin{verbatim}
NO
YES
YES
NO
NO
YES
\end{verbatim}

\textbf{\large Explanation:}

For the fourth case 209 can be reorder like this (029, 092, 209, 290, 902, 920), 029 is prime,
but is not a valid number because has leading zeros, in the valid numbers (209, 290, 902, 920).
there are no primes, so the answer is ``NO''

For the fith case 24 only has two possible numbers (24, 42) and there are no primes, so the
answer is ``NO''.

\newpage
        \begin{center}
            {\LARGE Problem E}\\
            {\Large Toby And Numbers}\\
            {Time limit: 2 second}\\
            {Memory limit: 256 megabytes}
        \end{center}\textbf{\large Problem Description}

Toby just invented a game with numbers, initially you have a list of numbers
and at any moment you can do the following operation:

\begin{itemize}
    \item Take two different numbers and replace the greater one with the absolute
    difference of those numbers.
    \item The game ends when all the numbers are equal.
\end{itemize}

It can be shown that the game always ends and no matter how you play, the result
will be always the same.

Help toby to determine what is the result for several instances of the game.

\textbf{\large Input Format}

The input begins with an integer $T$ denoting the number of test cases. For each
test case, there is a number $N$ indicating how many numbers are in the current case,
followed by $N$ positive integers.

$1 \le N \le 100$

Each one of the $N$ numbers is between 1 and 10000

\textbf{\large Output Format}

Print one integer for each test case. Note that at end of each game all the
numbers are equal, for this reason you only need to print it once, no matter what
is the size of the input list.

\textbf{\large Sample Input}

\begin{verbatim}
2
3
3 6 9
5
2 8 10 16 36
\end{verbatim}

\textbf{\large Sample Output}

\begin{verbatim}
3
2
\end{verbatim}

\newpage
    \end{document}
